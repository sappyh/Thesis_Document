\documentclass{kththesis}


\usepackage{blindtext} % This is just to get some nonsense text in this template, can be safely removed
\usepackage{graphicx}
\usepackage{tabularx}
\usepackage{multirow}
\usepackage{csquotes} % Recommended by biblatex
\usepackage[backend=biber]{biblatex}
\usepackage{listings}
\usepackage{algorithmicx}
\usepackage{algorithm}
\usepackage{algpseudocode}
\usepackage[counterclockwise, figuresleft]{rotating}

\addbibresource{stateofart.bib}
\addbibresource{USBMon.bib} % The file containing our references, in BibTeX format

\usepackage{amsmath}
\usepackage{mathtools}
\DeclarePairedDelimiter\ceil{\lceil}{\rceil}
\DeclarePairedDelimiter\floor{\lfloor}{\rfloor}


\usepackage{pdfpages}

\usepackage{acro}
\usepackage{longtable}

\DeclareAcronym{LPWAN}{
	short=LPWAN,
	long=Low Power Wide Area Network}
	
\DeclareAcronym{sdr}{
  short = SDR,
  long = Software Defined Radio
}

\DeclareAcronym{IO}{
  short = IO,
  long = Input Output
}

\DeclareAcronym{DMA}{
  short = DMA,
  long = Direct Memory Access
}

\DeclareAcronym{csma}{
  short = CSMA,
  long = Carrier Sense Multiple Access
}

\DeclareAcronym{mac}{
  short = MAC,
  long = Medium Access Control
}


\DeclareAcronym{tdma}{
  short = TDMA,
  long = Time Division Multiple Access
}

\DeclareAcronym{cpu}{
  short = CPU,
  long = Central Processing Unit
}

\DeclareAcronym{ack}{
  short = ACK,
  long = Acknowledgement
}

\DeclareAcronym{IOT}{
  short = IoT,
  long = Internet of Things
}

\DeclareAcronym{PHY}{
  short = PHY,
  long = Physical
}

\DeclareAcronym{OSI}{
  short = OSI,
  long = Open Systems Interconnection
}

\DeclareAcronym{RF}{
  short = RF,
  long = Radio Frequency
}

\DeclareAcronym{LQI}{
  short = LQI,
  long = Link Quality Information
}

\DeclareAcronym{FCS}{
  short = FCS,
  long = Frame Control Sequence
}

\DeclareAcronym{CSMA/CA}{
  short = CSMA/CA,
  long = Carrier-sense multiple access with collision avoidance
}

\DeclareAcronym{CSMA/CD}{
  short = CSMA/CD,
  long = Carrier-sense multiple access with collision detection
}

\DeclareAcronym{L2}{
  short = L2,
  long = Layer 2
}

\DeclareAcronym{FPGA}{
  short = FPGA,
  long = Field Programmable Gate Array
}

\DeclareAcronym{DSP}{
  short = DSP,
  long = Digital Signal Processor
}

\DeclareAcronym{NIC}{
  short = NIC,
  long = Network Interface Controller
}

\DeclareAcronym{FPRF}{
  short = FPRF,
  long = Field Programmable RF
}

\DeclareAcronym{UWB}{
  short = UWB,
  long = Ultra Wide Band
}

\DeclareAcronym{LNA}{
  short = LNA,
  long = Low noise amplifier
}

\DeclareAcronym{DDS}{
  short = DDS,
  long = Direct Digital Synthesis
}

\DeclareAcronym{NCO}{
  short = NCO,
  long = Numerically Controlled Oscillator
}


\DeclareAcronym{DAC}{
  short = DAC,
  long = Digital Analog Converter
}


\DeclareAcronym{FIR}{
  short = FIR,
  long = Finite Impulse Response
}

\DeclareAcronym{ASIC}{
  short = ASIC,
  long = Application Specific Integrated Circuit.
}

\DeclareAcronym{PDU}{
  short = PDU,
  long = Packet Data Unit.
}


\DeclareAcronym{MIMO}{
  short = MIMO,
  long = Multiple Input Multiple Output.
}

\DeclareAcronym{SPI}{
  short = SPI,
  long = Serial Peripheral Interface.
}

\DeclareAcronym{PGA}{
  short = PGA,
  long = Programmable Gain Amplifier.
}

\DeclareAcronym{PLL}{
  short = PLL,
  long = Phased Lock Loop
}

\DeclareAcronym{TSP}{
  short = TSP,
  long = Transreceiver Signal Processor}

\DeclareAcronym{IQ}{
  short = IQ,
  long = In-Phase Quadrature.
}

\DeclareAcronym{DDR}{
  short = DDR,
  long = Double Data Rate
}

\DeclareAcronym{FSM}{
  short = FSM,
  long = Finite State Machine
}

\DeclareAcronym{GPIF}{
  short = GPIF,
  long = General Programmable Interface
}

\DeclareAcronym{VHDL}{
 short = VHDL,
 long = VHSIC Hardware Description Language
}

\DeclareAcronym{RTL}{
 short=RTL,
 long= Register Transfer Level
}

\DeclareAcronym{FIFO}{
 short=FIFO,
 long= First In First Out
}

\DeclareAcronym{USB}{
 short=USB,
 long= Universal Serial Bus
}

\DeclareAcronym{I2C}{
 short=I2C,
 long= Inter-Intergrated Circuit
}

\DeclareAcronym{LR-WPAN}{
 short= LR-WPAN,
 long=Low-Rate Wireless Personal Area Network
}

\DeclareAcronym{6LoWPAN}{
 short= 6LoWPAN,
 long=IPV6 over Low-Power Wireless Personal Area Network
}

\DeclareAcronym{O-QPSK}{
 short= O-QPSK,
 long=Offset Quadrature Phase Shift Keying
}

\DeclareAcronym{BPSK}{
 short= BPSK,
 long= Binary Phase Shift Keying
}
\DeclareAcronym{ASK}{
 short= ASK,
 long= Amplitude Shift Keying
}
\DeclareAcronym{CSS}{
 short= CSS,
 long= Chirp Spread Spectrum
}

\DeclareAcronym{GTS}{
 short= GTS,
 long= Guaranteed Time Slot
}

\DeclareAcronym{PAN}{
 short= PAN,
 long= Personal Area Network
}

\DeclareAcronym{FFD}{
 short= FFD,
 long= Full Function Device
}

\DeclareAcronym{RFD}{
 short= RFD,
 long= Reduced Function Device
}

\DeclareAcronym{CRC}{
 short= CRC,
 long= Cyclic Redundancy Check
}

\DeclareAcronym{SHR}{
 short= SHR,
 long= Synchronization Header
}

\DeclareAcronym{SFD}{
 short= SFD,
 long= Start-of-Frame Delimiter
}

\DeclareAcronym{PN}{
 short= PN,
 long= Pseudo Noise
}

\DeclareAcronym{PCIe}{
 short = PCIe,
 long= Peripheral Component Interconnect Express
}

\DeclareAcronym{API}{
 short = API,
 long= Application Programming Interface
}

\DeclareAcronym{ARQ}{
 short = ARQ,
 long= Automatic Repeat reQuest
}

\DeclareAcronym{GPMC}{
 short = GPMC,
 long= General Purpose Memory Controller
}

\DeclareAcronym{UHD}{
 short = UHD,
 long= USRP Hardware Driver
}

\DeclareAcronym{RAT}{
 short = RAT,
 long= Radio Access Technology
}

\begin{document}

%\acsetup{first-long-format=\itshape}


% Frontmatter includes the titlepage, abstracts and table-of-contents
\frontmatter

\includepdf[pages={1}]{Chapters/cover.pdf}
%\titlepage

\setcounter{secnumdepth}{2}
\setcounter{tocdepth}{2}
\tableofcontents
\listoffigures
\listoftables

\clearpage

\acsetup{list-style=longtable, list-heading=chapter*}
\printacronyms[ heading= chapter*, sort=true]

%[
% name = {Abbreviations},
% sort = true,
%]

%\input{Chapters/acronym.tex}
\mainmatter

% Mainmatter is where the actual contents of the thesis goes


\chapter{Introduction}
The \ac{IOT} is enabling communication among huge numbers of diverse low power devices.  
According to an estimate by Ericsson \cite{noauthor_internet_2017}, there will be 20 billion connected \ac{IOT} devices by 2023.
Modern communication protocols need to evolve rapidly to enable reliable connection among these devices.
The communication needs for a field temperature sensor differs from those of an industrial controller. 
Hence, there is need for research and development of communication protocols that satisfy these diverse device communication needs. 
The evaluation of these experimental protocols is difficult because of the need of specialized radio hardware.
Simulation is widely used to evaluate these protocols but they fall short on modeling of real world performance.
\ac{sdr} devices can be a powerful platform for enabling the real-world evaluation of these protocols.\\

\ac{sdr} are flexible radio platforms where most of the communication systems functionality is designed in software. Typically, \ac{sdr} platforms have on board radio front-end equipped with wide band antennas and analog signal processing chain for tuning the carrier frequency and desired bandwidth. High speed data converters convert the incoming analog signals into the digital domain and vice-versa. In traditional radios, the digital processing chain of a wireless protocol physical layer is implemented on the same chip as the radio front-end and analog signal processing functions. \ac{sdr}, on the other hand, in host-PHY \cite{nychis_enabling_nodate} architecture transfers the converted data to a general purpose computing platform using bus transfer (USB, PCIe).  The digital processing chain is designed in software, thus allowing for flexibility in the protocol design, enabling experimentation in decoding and modulation techniques. \ac{sdr} also allows for careful analysis of RF signals as the raw sample data is made available to the host.\\

\begin{figure}[!h]
\centering
\includegraphics[width=0.75\textwidth]{Figure/SDRSystem.png}
\caption{Software Radio and Traditional Radio Architecture.}
\label{sdr_architecture}
\end{figure}


%\ac{sdr} helps to protect investments by facilitating change of protocols on already existing system. A major motivation within the commercial communications arena, is the rapid evolvement of communications standards, making software upgrades of base stations a more attractive solution than the costly replacement of base stations\cite{ulversoy_software_2010}. \ac{sdr} also opens up the possibility of Cognitive Radios, a context sensitive radio system that can adapt depending on the radio channel conditions and applications. \\

The movement of digital signal processing functions from hardware to software leads to performance issues in \ac{sdr} systems.
A fundamental challenge of \ac{sdr} system is computational horsepower, because it needs to process complex data wave-forms in a reasonable time-frame. Since \ac{sdr} involves transferring of signals and data from one system to another, this introduces considerable communication delays. Finally, general purpose processing systems introduces non-determinism in data processing and communication processes.\\


\section{Problem Context}
Wireless devices share the wireless channel with other devices. The wireless protocol \ac{mac} layer is responsible for moderating access to the wireless channel. It typically uses \ac{tdma} and \ac{csma} to allocate the use of the channel. \ac{tdma} protocols schedule the allocation of the entire channel to one of the devices for a particular time duration. This requires global time synchronization among the devices so that the devices can understand when to transmit and receive. \ac{csma}, on the other hand uses the channel on an opportunistic basis, with the devices sensing if the channel is free or not.\\

\subsection{CSMA}
As highlighted by \cite{schmid_experimental_2007}, \ac{sdr} based systems don't comply with the stringent timing constraints imposed by modern \ac{mac} protocols. Furthermore, the presence of long bus communication and processing delays create blind spots\cite{schmid_experimental_2007} in carrier sensing. In Figure \ref{blind_spots}, the \ac{sdr} system is receiving a packet being transmitted on the air medium. Since there is communication and processing delays, the \ac{cpu} of the \ac{sdr} system  receives the packet completely at $t_1$ delayed from $t_0$ when the packet transfer ends on the air medium.\\

Once the packet has been received, the system wants to let the transmitting system about the successful reception by sending the \ac{ack} packet. If it detects the medium is free using carrier sensing, it would start transmitting.
But because of the delays, it would make this decision based on past information which has been delayed by $t_1 - t_0$.
This makes the system blind towards the real time channel situation when making the decision to transmit and might lead to a collision. \\ 

\begin{figure}[!h]
\centering
\includegraphics[width=0.6\textwidth]{Figure/BlindSpots.png}
\caption{Blind Spots Illustration(adapted from \cite{schmid_experimental_2007}).}
\label{blind_spots}
\end{figure}

Hence when designing \ac{mac} protocols, these delays needs to be taken into account to avoid collisions.
This necessitates a closer understanding of these delays and how different system parameters affect these delays.

%\subsection{TDMA}
%\ac{tdma} based protocols are controlled by time slots, hence there is need for precise scheduling to ensure that the transmissions happen in the correct time-slot. The delays and imprecise scheduling can be tolerated by making the time-slots longer but that degrades the efficiency of the overall network. Modern contention based protocols(\ac{csma}) also require precise timing to implement inter-frame spacing.\\

%Hence methods to implement precise time scheduling needs to studied.

\section{Project Context}

The project was conducted at RISE SICS as part of the 5G-Coral project.
5G-Coral is an European Union H-2020 project which envisions a convergent radio access network.
The project envisions numerous small multi-\ac{RAT} gateway.
The radio-head should be flexible to handle traffic from different devices running different protocols to enable convergent access.
For the feasibility of this goal, the cost effectiveness of the radio-head needs to be taken into account.
LimeSDR \cite{noauthor_limesdr_nodate} provides a cost-effective \ac{sdr} platform, which supports the desired frequency bands making it the ideal choice as the project's radio-head.\\

Low power wireless devices are one of the main focus areas for the 5G-Coral project.
IEEE 802.15.4 is one of most popular the network specifications for \ac{LPWAN} i.e \ac{IOT} systems.
It specifically defines the Physical Layer and the \ac{mac} layer of the network stack.
So there is a need to identify the performance bottlenecks of \ac{sdr} implementation of IEEE 802.15.4 for the successful deployment of the 5G-Coral project.
As there are no previous studies on the LimeSDR platform, this project evaluates the timing bottlenecks.

%The lack of previous research on the characteristics of the platform made it the ideal choice for my \ac{sdr} platform.\\








\section{Research Question}
%\begin{itemize}
%\item{
Taking into consideration the problem and project context, I formulated this research question:
--- Need to work on this part
What are the timing delays in  LimeSDR based IEEE 802.15.4 network ?
%}
\begin{figure}[!h]
\centering
\includegraphics[width=0.75\textwidth]{Figure/RQ1.png}
\caption{Research Question}
\label{rq1}
\end{figure}

The research question is explained using Figure \ref{rq1}.
The Host Side processing delay is introduced by running the software implementation of the 802.15.4 \ac{PHY} and \ac{mac} layers.
The bus communication delay represents the delay caused by the \ac{USB} 3.0 bus transfers.
The \ac{sdr} processing delay is the time required by the LimeSDR platform for transmission or reception of radio signals.
The objective of this thesis is to quantitatively evaluate these delays and the impact of different network and \ac{sdr} configuration parameters.
%\item{ How to implement precise scheduling in LimeSDR based IEEE 802.15.4 communication system? }
%\end{itemize}

\section{Report Outline}
The remainder of the report is structured as follows. \textit{Chapter 2} introduces the relevant background information for understanding the rest of the report.
Relevant previous work will be introduced in \textit{Chapter 3}.
\textit{Chapter 4} introduces the experimental setup and the methods used in the measurement of the timing delays. \textit{Chapter 5} presents the experimental results and analysis.
The method used in mitigation of these delays will be introduced in \textit{Chapter 6}.
\textit{Chapter 7} presents the experimental results and analysis of the mitigation technique used.
Finally, \textit{Chapter 8} includes the concluding remarks and scope of future work.

\chapter{Background}
This chapter introduces the necessary background information needed for understanding the rest of the report.
First section, provides a broad introduction to \ac{sdr} systems, GNU Radio Software Tool and \ac{PHY} and \ac{mac} layers of the network stack.
The second section introduces previous work in this domain, which are critically analyzed to help concretely define the problem area.
Finally the last section, introduces the LimeSDR platform, IEEE 802.15.4 based system design and the tools used in the methods section.


\section{Essential Concepts}
\subsection{\ac{PHY} and \ac{mac} Layers}
\ac{OSI} Model (ISO/IEC 7498-1:1994) presents the abstract model for networking, that is used for most communication systems design.
The abstract model is divided into 7 layers,  where entity in each layer implements the functionality of the layer and interacts directly with the layer beneath it, the added functionality can be used by the upper layers.
Data from the user application is encapsulated by each subsequent layers of the \ac{OSI} model into their frame format.
These frames carry meta-data in the form of frame headers.
Different protocols use different frame format and headers  as it helps differentiate one protocol from another.
These headers help the receiver in learning where the incoming data packet is coming from, who is it meant for, how to decode and arrange the contents of the data packets etc.\\

\ac{PHY} layer is the lowest layer (L1) of the \ac{OSI} model,it interacts with the physical communication channel directly.
It defines the type of data transfer(serial/parallel) and data rate of the protocol.
\ac{PHY} layer defines the process of transmitting raw bits through the physical medium.
The bit-stream is grouped into code words and converted to symbols, which are then modulated to a physical signal for transmission over the transmission medium.
\ac{PHY} layers also provides physical transmission link information like carrier sense and collision detection and \ac{LQI}  to the upper layers. \\


\ac{mac} layer, \ac{L2} of the \ac{OSI} model, is responsible for defining the the methods for sharing and using the common transmission medium among multiple devices.
\ac{mac} layer addresses are used to to check if the incoming packet is meant for the device.
In case of outgoing packets, the \ac{mac} layer adds the \ac{mac} address of the destination device to the packet header.
It adds the synchronization preamble and \ac{FCS} for checking transmission error.
Retransmission in case of dropped packets and acknowledgement to successfully received packets are handled by this layer.\\

\ac{csma} is \ac{L2} protocol of the OSI Model. 
It is a method for handling multiple access of a shared medium.
It mainly comes in two varieties: \ac{CSMA/CD}  and \ac{CSMA/CA}. In the older \ac{CSMA/CD}, the nodes wait until the frame is ready, then check is the medium is idle or not.
If idle it starts transmission.
During transmission, it monitors the medium for collision.
If collision is detected, it employs a collision recovery process, where it sends a jam signal to signal other nodes that a collision has occurred.
Then it waits for a  random delay and starts transmission again.\\

\ac{CSMA/CA} tries to avoid collision, it starts off similar to \ac{CSMA/CD} where it senses to check when the channel is idle.
If found idle, it starts transmission.
As it is difficult for wireless nodes to detect collision at the same time its transmitting therefore it relies on an \ac{ack} from the receiving node to check if the data packet was received.
If \ac{ack} is not received, the node assumes a collision has occurred and  uses exponential back-off to determine when the next time to re initiate transmission.\\
\begin{figure}[h!]
\label{Csma_flow}
\centering
\includegraphics[width=0.9\textwidth]{Figure/CSMA.png}
\caption{CSMA flow graph.}
\end{figure}

\ac{tdma} is also a \ac{L2} protocol, where a coordinator schedules medium access to the nodes in a periodic manner.
Communication happens in time-slots.
Each node in the network is given exclusive access to transmit during its time slot.
The coordinator generates beacon signals periodically to maintain relative time synchronization. On receiving the beacons, the nodes adjust their transmit clocks so that they have the correct estimate of their time-slots.
 
 

\subsection{\ac{sdr} Platforms}

\ac{sdr} represents a new paradigm of communication system design where the system is flexible to adapt to the needs of the end-user as also the radio channel conditions. Nychis et.al \cite{nychis_enabling_nodate} classifies \ac{sdr} based communication systems into two main architectures.
\begin{itemize} 
\item{\textit{Host-PHY Architecture:} This is the most common architecture, enabling design and development of the entire system in software.
It provides the maximum flexibility in terms of design and implementation choices, also there is added benefit of easy upgrades.
But, since the system is designed is software only, the processing and communication delays make most modern \ac{mac} protocols infeasible in this architecture.}


\item{\textit{NIC-PHY Architecture:} In this architecture most of the \ac{PHY} layer functionality is implemented in \ac{FPGA} and \ac{DSP}.
The closer proximity to the radio hardware and specialized parallel hardware processing makes this architecture most suitable for running the modern \ac{mac} protocols.
But the design process for this architecture based systems is time consuming and difficult, as traditionally hardware programming harder simple software programming.
However, they are much more flexible compared to commercial \ac{NIC}.
\textit{Wireless Open Access Research Platform}(WARP) \cite{noauthor_warp_nodate} is an example of system based on this type of architecture.}

\end{itemize}
Since host-PHY \cite{nychis_enabling_nodate} is the most commonly used architecture, the report concentrates on explaining the functionality of \ac{sdr} platforms using this architecture.
Figure \ref{host_PHY} shows the typical design of communication systems in this architecture.


\begin{figure}[h!]
\centering
\label{host_PHY}
\includegraphics[width=\textwidth]{Figure/Host_Phy.png}
\caption{Host-PHY \cite{nychis_enabling_nodate} \ac{sdr} architecture.}
\end{figure}

\subsection{GNU Radio}

\section{State of the Art}
\section{Base System and Tools}
 
\subsection{LimeSDR-USB}
The LimeSDR-USB uses a USB 3.0 interface for communicating with the host computer. It supports MIMO operations with 2 RX and TX channels operating simultaneously. The maximum sampling rate supported by the ADC and DAC of LMS7002M is 160 Mhz, but the USB 3.0 restricts it to 61.44 MSPS when all the RX and TX channels are used simultaneously.

\subsubsection{LimeSDR USB dataflow.}
\begin{figure}[h!]
\centering
\includegraphics[scale=0.6]{Figure/Software_Architecture.jpg}
\caption{LimeSDR USB software architecture}
\end{figure}

\begin{itemize}
\item{\textit{TX Data Path:} Stream data from GNURadio is passed through Soapy drivers to the LimeSDR Streaming API. The API unwraps the data and the control flags and pushes it to the TX-FIFO. It also does the necessary data representation translation depending on the data format of the streamed data. For example, in case of complex data, it changes 32 bit I and Q value representation of GNU Radio to 16 bit I and Q values representation.  The values are pushed to the respective TX stream channel buffers(TXFIFO). The connection stream initializes the TX buffers and fills them with data from the TXFIFO. The Write function structures the data into FPGA data packet structure((Figure \ref{fpga_packet})) and combines multiple such packets into predefined batch size(initially 4). The buffer is processed by libusb to create bulk transfer packet and finally streamed to output data endpoint.}

\item{\textit{RX Data Path: } The USB data is continuously streamed from the LimeSDR to the Connection stream buffers through libusb. The Read function waits for data to be available from a usb context for the endpoint it is listening to, then it transfer data from the endpoint, parses the FPGA packets (Figure \ref{fpga_packet}) to collect the data and pushes them to the RXFIFO. If the rx stream is configured to have particular receive time, it checks if that condition is satisfied. The LimeSDR streaming API collects the data from the RXFIFO and does the necessary data interpretation translation (reverse translation to the TX Data Path), finally streams the data to GNURadio. }
\end{itemize}

\subsubsection{LimeSDR USB packets and endpoints}
LimeSDR uses four different endpoints for USB data transfer, these endpoints as Data Endpoint and Control Endpoint for both input and output directions. The control endpoints are used for configuring and retrieving data from the LMS7002M and NIOS Core on the FPGA. Data packets are used for the streaming data. 
\begin{table}[h!]
\centering
\begin{tabular}{|c|c|}
\hline
Endpoint No. & Function\\
\hline
0x01 & Stream Data Output\\
0x81 & Stream Data Input\\
0x0F & Control Data Output\\
0x8F & Control Data Input\\
\hline
\end{tabular}
\caption{LimeSDR USB transfer endpoints}
\end{table}
It uses two different packet structures for the LMS7002M Control Packets and the Stream Data Packets. Depending on the control command, different number of bytes are packed into one data element and the maximum number of blocks in a single packet is defined. One LMS64C protocol packet (Figure \ref{lms_packet}) is maximum 64 bytes, if the data to be sent is larger than that then the data field is segmented into several packets. The block count gives the number of data element in a single packet. The FPGA contains 4080 bytes of data along with 8 bytes of counter data that can be used for timestamp on the TX packets. The Lime driver uses synchronous bulk transfer for LMS Control packets and asynchronous bulk transfers for the FPGA packets.
\begin{figure}[h!]
\centering
\includegraphics[width=\textwidth]{Figure/LMS64C_Packet.png}
\caption{LMS Control Packet Structure}
\label{lms_packet}
\end{figure}

\begin{figure}[h!]
\centering
\includegraphics[scale=0.6]{Figure/FPGA_Packet.png}
\caption{FPGA Packet Structure}
\label{fpga_packet}
\end{figure}




\subsection{USBMon}
It is kernel facility provided to collect I/O traces on the USB Bus\cite{_usbmon}. USBMon reports the requests made to and by the USB Host Controller Drivers(HCD). It provides two kinds of API's : binary and character. The binary API is accessed by character devices located in the /dev namespace. The character API provides human readability and uniform format for the traces.The kernel data from the USBMon text data is made available to the userspace using debugfs\cite{_debugfs} utility.

\begin{figure}[h!]
\centering
\includegraphics[width=\textwidth]{Figure/USBMon.png}
\caption{USBMon Architecture(Adapted from \cite{basak_usb_2018}).}
\end{figure}

\subsubsection{Text Data Format}
\begin{table}
\centering
\begin{tabular}{|c|c|c|c|c|}
\hline
URB Tag & Timestamp & Event Type & Address & URB Status\\
\hline
ffff8fbdbbae4000 & 2942307806 & S & Bo:3:008:15 & -115\\ 
\hline
\end{tabular}
\begin{tabular}{|c|c|c|}
\hline
Data Length & Data Tag & Data\\
\hline
64 & = & 21000100 00000000 002a0484 00000000 000000\\
\hline
\end{tabular}
\caption{Text USB Trace Example.}
\end{table}

\begin{itemize}
\item {\textit{URB Tag:} URB Identification number, it is usually the in kernel adress of the URB structure.}
\item{\textit{Timestamp:} The timestamp for the URB event at the HCD in microseconds. It is measured by the usbmon main utility using \textit{gettimeofday()} function of \textit{time.h}.}
\item{\textit{Event Type:} It specifies the event type of the HCD event. S - Submission C -Complete E - submission error.}
\item{\textit{Address: } It consists of four fields separated by colons. The URB type and direction, bus number, device number, endpoint number. The URB type and direction specifies the type of USB transfer(can be both synchronous and asynchronous).\\
\begin{table}[h!]
\centering
\begin{tabular}{|c|c|c|}
\hline
Bi & Bo & Bulk Input and Output.\\
Ci & Co & Control Input and Output.\\
Ii & Io & Interrupt Input and Output.\\
Zi & Zo & Isochronous Input and Output.\\
\hline
\end{tabular}
\caption{URB Type and Direction.}
\end{table}\\
The USB device transfers data through a pipe to a memory buffer on the host and endpoint on the device. The type of data transfer depends on the endpoint and the requirements of the function. The transfer types are as follows\cite{_usb_data_transfer}:

\begin{itemize}
\item{\textbf{Control Transfers:} It is mainly used for configuration, command and status operations.}
\item{\textbf{Bulk Transfers:} Bulk Transfer are used for bulky,non-periodic non time-sensitive burst transmissions.}
\item{\textbf{Interrupt Transfers:} It is used for mainly sending small amounts of data infrequently or asynchronously.}
\item{\textbf{Isochronous Transfers:} Isochronous transfers are mainly used for periodic, continuous streams of time sensitive data.} 
\end{itemize}
USB endpoint as explained by \cite{_usb_endpoint} , refers to the buffers on the USB device. The host computer irrespective of the host operating system can communicate by reading and writing to these buffers. They can be data endpoints and control endpoints.Data endpoints are used for transferring data whereas the control endpoint is used for configuration and device specific control.
}

\item{\textit{Data Length:} For urb\_submit it gives the requested data length and for callbacks it is the actual data length.}

\item{\textit{Data tag:} If this field is '=' then data words are present.}

\item{\textit{Data:} The data words contains in the USB transfer packet.}
\end{itemize}

\subsubsection{Raw Binary}
The overall data format is same as the text data, the data is available in raw binary by accessing character devices at /dev/usbmonX. The data can be read by using \textit{read} with \textit{ioctl} or by mapping the buffer using \textit{mmap}. The usbmon events are buffered in the following format:

\begingroup
\centering\scriptsize\begin{lstlisting}
struct usbmon_packet {
	u64 id;			/*  0: URB ID - from submission to callback */
	unsigned char type;	/*  8: Same as text; extensible. */
	unsigned char xfer_type; /*    ISO (0), Intr, Control, Bulk (3) */
	unsigned char epnum;	/*     Endpoint number and transfer direction */
	unsigned char devnum;	/*     Device address */
	u16 busnum;		/* 12: Bus number */
	char flag_setup;	/* 14: Same as text */
	char flag_data;		/* 15: Same as text; Binary zero is OK. */
	s64 ts_sec;		/* 16: gettimeofday */
	s32 ts_usec;		/* 24: gettimeofday */
	int status;		/* 28: */
	unsigned int length;	/* 32: Length of data (submitted or actual) */
	unsigned int len_cap;	/* 36: Delivered length */
	union {			/* 40: */
		unsigned char setup[SETUP_LEN];	/* Only for Control S-type */
		struct iso_rec {		/* Only for ISO */
			int error_count;
			int numdesc;
		} iso;
	} s;
	int interval;		/* 48: Only for Interrupt and ISO */
	int start_frame;	/* 52: For ISO */
	unsigned int xfer_flags; /* 56: copy of URB's transfer_flags */
	unsigned int ndesc;	/* 60: Actual number of ISO descriptors */
};	
\end{lstlisting}
\endgroup


\subsection{pidstat}
\subsection{802.15.4}
\chapter{Literature Study}

This chapter introduces previous research work on the research question.
As there is no prior work done on the LimeSDR-USB platform, all the research work investigated will be using USRP as the \ac{sdr} platform.
The previous work is analyzed to figure out the methods used for measurement of the timing delays.
The results of each of these works are analyzed to figure out the performance bottlenecks.
Finally, previous work is explored for mitigation strategies.\\

Schmid et al \cite{schmid_experimental_2007} focused on characterizing the latency and its impact on throughput for modern protocols like IEEE 802.15.4 in a Host-PHY architecture.
The work introduced the problem of blind spots and the impact of bus transfer latency.
It used an external oscilloscope with one channel connected to the parallel port of the host computer and the other channel connected to one of the RF ports of the USRP as the measurement setup.
It concentrated on analytically modelling the bus latency and assumed the rest of the delay is introduced by the software processing.
%In their measurements they ignore the software processing delays and attribute their experimental results to only bus latency.
%But, the software processing delays is found to be quite significant in their later experiments which leads me to believe that the reported bus latency delays are higher than the actual values.
A key takeaway from the results is USB transfer time depends on the USRP buffer size specified at system initialization.
The bus latency is significant at lower sampling rates as it takes more time to fill up the USRP buffers.
But, at higher sampling rates, the bus latency is negligible compared to the processing delay.
% The work measured round trip times but described neither the method nor the measurement setup.
\\

Nychis et al \cite{nychis_enabling_nodate} argued for the need for split-MAC approach, where time-sensitive \ac{mac} operations are moved to the \ac{FPGA} which are controlled by the host computer.
%They refer to this design as split-MAC approach to \ac{sdr} system design.
%LimeSDR has already implemented data packet meta-data, control channel and precise scheduling suggested by Nychis et al.
Since this project concentrates on timing delay characterization, the discussion is only limited to the relevant delay analysis.\\

The Split-MAC approach was motivated by precise time information of delays at two levels: kernel and user space, kernel and FPGA.
Timestamps were introduced at different points of the TX and RX Chains to quantitatively measure these delays.
For the Kernel to FPGA time, the work modified the kernel's USB driver and measured the time at the last point before the DMA Write Request or after DMA read request interrupts the driver.
USRP ping command was used for the  measurement of the overall round trip time.
It is important to note that this measurement setup did not use the radio frontend of the USRP.
So the reported bus transfer time is not controlled by the sampling rate as shown by Schmid et al \cite{schmid_experimental_2007}.
Even in that case, the bus transfer time was quite significant.
The work further modified the USB transfer size to 512 bytes from the default 4096 bytes.
The work concluded that USB transfer setup time is quite significant as this modification led to reduction of Kernel to FPGA only by a factor of two.
Although, kernel to FPGA time contributed significantly to the overall latency, it contributed a limited amount of jitter in their results.
On the other hand, GNU Radio processing had high standard deviation.\\

Truong et al \cite{truong_investigating_2013} investigated and analyzed the different sources of delays in an USRP Embedded E 100 \ac{sdr} platform.
The USRP E series has an embedded processor which allows it to operate it in a standalone mode.
Instead of communicating with a host computer through a communication bus, the E series uses a  \ac{GPMC} controller for connecting the embedded memory and the \ac{FPGA} buffers.
The measured latency was segmented into three parts: software, bus and hardware delays.
Software delay is defined as the delay introduced by the software buffering scheme in GNU Radio and other host computer processes involved.
Bus Delay is the delay introduced by the buffering in the \ac{UHD}.
The hardware delays is caused mainly by buffering of data in the FPGA \ac{FIFO} buffers which is proportional to the USRP sample rate.
Similar to Nychis et al \cite{nychis_enabling_nodate}, this work measured overall round trip times as well as individual component delays using timestamps at different steps.
% The timestamps were similar to the ones used by Nychis et.al \cite{nychis_enabling_nodate}.
They used ping command to evaluate the overall latency.
A GNU Radio flow-graph was used for the timestamp method.
One thing to note is that the results are computed over 873 out of 5000 ping messages sent.   
Finally, the work showed the impact of \ac{UHD} buffer size on latency, with lower buffer size leading to lower mean and standard deviation in the measured latency.
%Although the works stated it takes timestamps at different steps, the results for those were not presented completely.
The work concluded host computer processing time is the main latency bottleneck as they are using an embedded processor. \\




%Surligas et al works on providing a \ac{sdr} platform for using heterogeneous wireless standards: 802.11 and 802.15.4.
%The work provides a host-PHY architecture and extends the existing UCLA PHY with Acknowledgements and Channel Sensing.
%They showed their implementation of 802.15.4 is able to exchange packets with Zolertia Z1 platform, but they do not concentrate on the delays inherent in this process.\\
 
Puschmann et al \cite{puschmann_implementation_2011} developed a Send-and Wait \ac{ARQ} protocol  testbed using USRP 2 as the \ac{sdr} platform.
They evaluated the testbed by measuring end to end throughput and latency.
In this work, ping command was used for measuring the round trip times at data-link layer and the application layer.
They investigated the impact of sample buffer size in the USRP2 driver on the round trip times.
Lower sample buffer sizes led to lower round trip times with less jitter as the received samples do not need to wait unnecessarily in the queues.\\

As highlighted by all these previous works, there is significant latency in \ac{sdr} platform based communication systems.
Previous works have tried to showcase delays in different segments in the processing chain but comprehensive evaluation is missing.
The literature review showed that there is lack of knowledge about latency of recent state of the art platforms like LimeSDR.
One of the purposes of this project is to fulfill this knowledge gap.
Finally, most of the works tried to mitigate the buffering delays in the \ac{UHD} by modifying the UHD sample buffer size.
Since, LimeSDR implements a different software architecture alternative mitigation techniques need to explored. 



\chapter{Methods}
This chapter introduces the methodology and system architecture followed by the experimental designs for the quantitative analysis.
The system architecture describes the GNU Radio flow-graph description, software and hardware used in this project. The experiments are presented in chronological order. In the first experiment, the performance of the system is measured with respect to broader parameters like sampling rate, data payload size and the number of message sent per second. The second experiment was designed evaluation of the impact of these parameters on the the different subsections of the data path. In experiment 3, the project looks at the impact of the size of the bus transfer on the overall delay.

\section{Methodology}
This project use quantitative experimental design method for the purpose of this project.
The characterization of timing delays needs to have quantitative measurements as they would help compare it to the specified standards and define the limitations concretely.
This project designs experiments to collect these data as well as have a causal effect of system parameters on the measurements.\\

A method was needed for measuring the time duration of the individual process as this project aims for timing characterization of signal processing chains.
The LimeSDR doesn't have a real time clock on board, so in order for time measurement on the LimeSDR, it was needed to synchronize to synchronize different clock domains across a variable latency link.
This process is complex and not feasible given the time constraint, so this project avoids doing any time measurement on the LimeSDR.
Taking hints from the literature review, this project opts to timestamp the \ac{sdr} software processes on the host computer.
This method is reliable and doesn't introduce affect the measurements significantly.\\

Most of the previous studies used ping as the measurement tool.
In this case, there are two host computer connected through two \ac{sdr} platforms.
As the host computers will not be identical they will impact the collected measurement differently.
It would not be possible to evaluate the impact of the host computer processing resources in this case.
Another problem with this method is because this uses common frequency bands, communication among other devices can make the setup unreliable.
As this project concentrates on the performance of the LimeSDR platform without considering the channel conditions, the impact of communication through the air medium can be safely ignored.
Finally, measurement across two different platforms makes it difficult to correlate the measurements on the two computer as the clocks are not synchronized.
Considering these factors, this project picked the loopback method.
In this method, the TX and RX chains on the LimeSDR are shorted before the \ac{LNA} of the LMS7002M chip shown in Figure \ref{lms7002m}.
This method uses the radio frontend of the LimeSDR which helps evaluate the impact of radio-frontend configuration on the performance.




\section{System Architecture}
With the methodology in place, the measurement setup needs to be described.
The measurement setup looks like 

The experimental system is primarily based on the WIME project implementation of 802.15.4 protocol. It is adapted to use the LimeSDR board instead of USRP as the \ac{sdr} platform. The adaptations can be grouped as

\begin{enumerate}
\item{GNU Radio blocks}
\item{802.15.4 \ac{PHY} layer}
\item{Periodic Message Source}
\end{enumerate}

\paragraph{GNU Radio Blocks}

For using the LimeSDR platform the USRP Sink and Source blocks are replaced by the grlimesdr project source and sink blocks. Another alternate used in this project is the gr-osmosdr project sink and source blocks which used soapysdr to access the Lime API. The former was chosen as it directly interacts with the LimeAPI without using the adaptation layer presented by the soapysdr project. This gives much better control of the board control parameters and also saves subsequent memcpy operations used by the soapysdr glue layer. (\textbf{Maybe present the RTT values for using the two different blocks and show that the limesdr block is better.}

)

\paragraph{802.15.4 \ac{PHY} layer}

The WIME project \ac{PHY} layer has been designed to only work with 4 MHz as the sampling rate, in this project, the \ac{PHY} layer has been modified to accommodate different sampling rates. \textbf{Describe the change}

\textbf{Describe the timing probe}
\paragraph{Periodic Message Source}

A periodic message source block was implemented in the GNU Radio, it takes in the message length and time period as parameters. Figure \ref{message_source} shows the working of the message source with respect to time. The data length controls the duty cycle of the signal by varying $\Delta T_{tx}$, which is the time is requires to transmit the message through USB.\\

The block notes down the global system time as $T_1$ when it publishes a message to its output port. When the block receives the message the time $T_1$ is written to a file for analysis with the reception time measured in the \ac{PHY} layer. As only valid messages are received by this block writing $T_1$ only on valid receive helps in measuring the time delay only for valid data points.\\

Since the time noted should be compared with those from usbmon, \textit{gettimeofday} was selected as the preferred method. The time period was set such that the transmitted message is received before sending the next message.\\

\begin{figure}[h!]
\centering
\includegraphics[width=\textwidth]{Figure/Message_Source.png}
\caption{Periodic Message Source}
\label{message_source}
\end{figure}


\subsection{System Description}

\begin{figure}[h!]
\centering
\includegraphics[width=0.8\textwidth]{Figure/Setup2.png}
\caption{System Description}
\label{setup_overview}
\end{figure}

\subsection{Software Description}
\subsection{Hardware Description}
\section{Timing Analysis}

The project uses the Wime Project implementation of 802.15.4 MAC and PHY layers in GNU Radio. For the purpose of measurement of round trip latency, a loop back experimentation setup(Figure \ref{setup_overview}) was implemented. A periodic message source generates messages and notes down the time T1. It is then processed and modulated by the 802.15.4 MAC and PHY respectively and sent through the OSMOCOM transreceiver to the LimeSDR. The RX and TX ports of the LMS7002M has been shorted and hence the original sent message loopbacks through the FPGA and comes back to the GNU Radio and is demodulated and processed by the PHY and MAC blocks respectively and is ultimately received by the periodic message source and the time is noted as T4.

The usbmon kernel utility continuously monitors bus activity between the LimeSDR USB driver and USB Host Controller. It timestamps the transfers and generates event queues to be accessed from the user space. The timing measurements program parses the event queue to find the relevant packets and notes down their usbmon timestamps as T3 and T4 for transmit and receive packets respectively.  
\subsection{Message Source} \label{message_source}

\subsection{Timing Measurements Program}
The timing measurements program uses ioctl to access the /dev/usbmonX character device. This allows the program to access the usbmon kernel utility event queue. The events are filtered to find packets with \textbf{0x01} \& \textbf{0x81} device endpoints. The data streams are parsed to find the relevant data fields from FPGA packets(Figure \ref{fpga_packet}), following that the data is converted from integer representation to complex floating point representation. The modulus of In Phase Sample's amplitude is used to determine if the data contained in the packet is useful or not. 

\begin{table}
\centering
\begin{tabular}{|c|c|}
\hline
USB Transfer Direction & Threshold Value  \\
\hline
|I| TX & 0.8\\
|I| RX & 0.2\\
\hline
\end{tabular}
\caption{Transfer Direction and Threshold Value}
\label{thres_table}
\end{table}

Analyzing the samples in the data stream, the samples threshold for actual data packets to as shown in Table \ref{thres_table}.  Once the packets have been analyzed, the sequence of events was studied to generate a state machine representation for the timing functionality.\\ 
The sequence follows the structure shown in figure \ref{sequence} if the condition mentioned about the time period in \ref{message_source} is satisfied. Since we want to measure the round trip delay, the time instant of the first TX and last RX packet as noted by T2 and T3 respectively needs to be measured. The difference between them gives the Kernel round-trip delay as measured by usbmon.    
\begin{figure}[h!]
\centering
\includegraphics[width=\textwidth]{Figure/Sequence.png}
\caption{Sequence of valid data packet with time}
\label{sequence}
\end{figure}
\begin{figure}[h!]
\centering
\includegraphics[scale=0.5]{Figure/State_Machine.png}
\caption{State Machine}
\label{state_machine}
\end{figure}

State Machine shown in Figure \ref{state_machine} controls the timing measurement function. It starts with state S0 and when it receives a TX event it sets T2 and moves to S1, further TX events don't update the value of T2 as we want the first TX event time. The state machine moves from S1 to S2 on a RX event, it sets the value of T3, further RX events updates the value of T3 as we want the time instant of the last RX event. On receiving TX event when at S2, it moves to S1, calculates $T3-T2$ and sets the value of T2.
\subsection{Results Correlation Method}
All the time instants are stored in Unix Time Format, a python script stores the values in separate arrays t1, t2, t3, t4 for GNU Radio Transmit Time, Kernel Transmit Time, Kernel Receive Time and GNU Radio Receive Time respectively. 

\begin{algorithm}[!h]
\caption{Time Data Correlation}
\begin{algorithmic}
\State {$T \gets $ Time Period of Message Source }
\State {$l \gets $ min(length of time arrays)}
\State $i \gets 0$
\For {$i < l $}
\If{($ t1[i]> t2[i]$ or $ t3[i]> t4[i]$)} \\
\hspace{1.35cm} delete $t1[i],t4[i]$
\ElsIf{$(t2[i]-t1[i]) > T $}
delete t2[i]
\ElsIf{$(t4[i]-t3[i]) > T $}
delete t3[i] 
\Else{ $i \gets i+1$ \\ \hspace{1.35cm} $l \gets $ min(length of time arrays) }
\EndIf
\EndFor
\end{algorithmic}
\end{algorithm}

Once the arrays have been compared to remove corrupt data, the mean and standard deviation of the respective arrays are found.




\chapter{Results and Analysis}
\subsection{Analytical Method}
The 802.15.4 PHY layer expands 1 byte of message data to 128 bytes, so the maximum packet length of 127 bytes becomes produces sample data of size
$127*128=16256 bytes=15.875KB$\\. The FPGA packet format adds 16 bytes overhead for every 4080 bytes so the overhead for 16256 bytes would be 64bytes. So the overall transfer size would be 16320 bytes. This would require four FPGA packets so the actual size of the USB transfer would be 16384 bytes
Now for sampling rate of 1MHz $\equiv$ 1MSPS, the actual data transfer is 1.5 MBps since the LMS7002M has 12 bits ADC and DAC. 
\begin{table}[!h]
\centering
\begin{tabular}{|c|c|}
\hline
Sampling Rate & USB Transfer delay \\
\hline
5 MHz & 4369.07 $\mu$s\\
10 MHz & 2184.53 $\mu$s\\
15 MHz & 1456.35 $\mu$s\\
20 MHz & 1092.27 $\mu$s\\
\hline
\end{tabular}
\caption{Analytical USB Transfer Delay}
\label{back_env}
\end{table}

\subsection{Experimental Results}
\begin{figure}
\centering
\includegraphics[scale=0.5]{Figure/results_setup.png}
\caption{Results Setup}
\label{res_set}
\end{figure}
Figure \ref{res_set} shows the different terminology used in the results, with the TX \& RX software delay is the delay caused by the GNU Radio and LimeSDR driver processing, the Kernel RTT Time includes the buffer delay in the LimeSDR and the USB communication delay. Total RTT Time = Kernel RTT Time + TX Software Delay + RX Software Delay. All the timing measurements are done on Lenovo Thinpad X240 with Dual-Core Intel® Core™ i5-4300U CPU @ 1.90GHz and 4GB RAM. The setup use Limesuite version 17.12.0 and gateware version 2.12.
\begin{table}
\centering
\label{res}
\begin{tabular}{|l|c|c|c|c|}
\hline
Sampling Rate(MHz)                                                  & 5    & 10   & 15   & 20   \\
\hline
Total RTT mean ($\mu$s)                                                      & 5360 & 3606 & 3065 & 4485 \\ \hline
Total RTT std deviation ($\mu$s)   & 326  & 472  & 262  & 1273 \\ \hline
Kernel RTT mean ($\mu$s)                                                     & 4113 & 1937 & 1354 & 762  \\ \hline
Kernel RTT std deviation ($\mu$s) & 1201 & 330  & 232  & 298  \\ \hline
TX chain mean ($\mu$s)                                                       & 470  & 675  & 831  & 1586 \\ \hline
TX chain std deviation ($\mu$s)    & 60   & 1165 & 186  & 860  \\ \hline
RX chain mean ($\mu$s)                                                       & 1100 & 1122 & 871  & 1769 \\ \hline
RX chain std deviation ($\mu$s)                                              & 472  & 1764 & 262  & 1036\\ \hline
\end{tabular}
\caption{Experimental results}
\end{table}



\subsection{Analysis}
\begin{itemize}
\item {The results show a monotonic drop in the kernel USB timings with increase in sampling rate and monotonic increase for RX and TX delay (Exception: 15 MHz). This indicates with increase in sampling rate, the buffers are getting overloaded and hence an increase in processing delay compared to bus communication delay.}
\item {Another thing that I noticed was at high sampling rate the round trip time increases with time, again pointing to buffer delay on the RX chain.}
\item {My measurement program becomes highly unstable at higher sampling rates, for example for 20MHz, I captured 610 packets of which I could correlate only 160 packets. This is mainly because the usbmon event queues overflow and hence my timing measurement program misses some relevant events and reports wrong timing information. One method I plan on using is flushing the buffers before the message source generates the message since each measurement is independent of the previous in a TDMA protocol.  }
\item {The analytical values for the RTT time( Table \ref{back_env}) is more than the actual value that usbmon reported(Table \ref{res}), my hypothesis is that this happens due to my assumption that all the data in the LimeSDR TX buffer is popped before the relevant RX data is popped back in the RX buffers on the LimeSDR side. But in actual operation even before all the TX data has been popped, the loopback data is being pushed to the RX buffers. This is demonstrated using figure \ref{staggering} where its shows the state of the RX and TX buffers with respect to time. \textit{t1} shows the instant when all the relevant data has been popped from the TX buffers and \textit{t2} shows the instant when the relevant RX data is popped from the buffers. For my assumption \textit{t1} should be equal to \textit{t2}, but here  $t2<t1$ hence the reported values are less than those of the analytical model. With increase in sampling rate the difference between t2 and t1 increases. There is a need to address this issue to ensure reliability of the measurement method.
\begin{figure}
\centering
\hspace{1cm}\includegraphics[scale=0.5]{Figure/Staggering.png}
\caption{Overlapping of buffers on LimeSDR}
\label{staggering}
\end{figure}
}
\end{itemize}

\printbibliography[heading=bibintoc] % Print the bibliography (and make it appear in the table of contents)

\appendix

% \chapter{Unnecessary Appended Material}

\includepdf[pages={2}]{Chapters/cover.pdf}
\end{document}
