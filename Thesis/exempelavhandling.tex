\documentclass[swedish]{kththesis}

\usepackage{blindtext} % This is just to get some nonsense text in this template, can be safely removed

\usepackage{csquotes} % Recommended by biblatex
\usepackage{biblatex}
\addbibresource{references.bib} % The file containing our references, in BibTeX format


\title{Detta är den svenska titeln}
\alttitle{This is the English translation of the title}
\author{Osquar Student}
\email{osquar@kth.se}
\supervisor{Lotta Larsson}
\examiner{Lennart Bladgren}
\principal{Företaget AB}
\programme{Civilingenjör Datateknik}
\school{Skolan för Datavetenskap och Kommunikation}
\date{\today}


\begin{document}

% Frontmatter includes the titlepage, abstracts and table-of-contents
\frontmatter

\titlepage

\begin{abstract}
  Svensk sammanfattning.

  \blindtext
\end{abstract}


\begin{otherlanguage}{english}
  \begin{abstract}
    English abstract.

    \blindtext
  \end{abstract}
\end{otherlanguage}


\tableofcontents


% Mainmatter is where the actual contents of the thesis goes
\mainmatter


\chapter{Introduktion}

Vi använder paketet \emph{biblatex} för litteraturreferenser.  Därför
anropar vi kommandot \texttt{parencite} för att få referenser inom
parentes, så här \parencite{heisenberg2015}.  Det är också möjligt
att använda författarens namn som en del av en mening genom att
använda \texttt{textcite}, om vi t.ex.\ talar om en studie av
\textcite{einstein2016}.

\Blindtext

\section{Frågeställning}

\blindtext

\chapter{Metodval}

\blindtext

\printbibliography[heading=bibintoc] % Print the bibliography (and make it appear in the table of contents)

\appendix

\chapter{Extra Material som Bilaga}

\end{document}
