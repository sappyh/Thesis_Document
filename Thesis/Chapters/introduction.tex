\chapter{Introduction}
\ac{IOT} is enabling communication among huge numbers of diverse low power devices.  
According to estimate by Ericsson \cite{noauthor_internet_2017}, there will be 20 billion connected \ac{IOT} devices by 2023 . 
The communication needs for a field temperature sensor is different from a industrial controller. 
Hence, there is need for research and development of communication protocols satisfying diverse device needs. 
The evaluation of these experimental protocols is difficult because of the need of specialized radio hardware. 
\ac{sdr} devices can be a powerful platform for enabling the real-world evaluation of these protocols.\\

\ac{sdr} are flexible radio platforms where most of the communication systems functionality is designed in software. Typically, \ac{sdr} platforms have on board radio front-end equipped with wide band antennas and analog signal processing chain for tuning the carrier frequency and desired bandwidth. High speed data converters convert the incoming analog signals into the digital domain and vice-versa. In traditional radios, the digital processing chain of a wireless protocol physical layer is implemented on the same chip as the radio front-end and analog signal processing functions. \ac{sdr}, on the other hand, in \textit{host-PHY(\cite{schmid_experimental_2007})} architecture transfers the converted data to a general purpose computing platform using bus transfer (USB, PCIe).  The digital processing chain is designed in software, thus allowing for flexibility in the protocol design, enabling experimentation in decoding and modulation techniques. \ac{sdr} also allows for careful analysis of RF signals as the raw sample data is made available to the host.\\

\begin{figure}[!h]
\centering
\includegraphics[width=0.75\textwidth]{Figure/SDRSystem.png}
\caption{Software Radio and Traditional Radio Architecture.}
\label{sdr_architecture}
\end{figure}


\ac{sdr} helps to protect investments by facilitating change of protocols on already existing system. A major motivation within the commercial communications arena, is the rapid evolvement of communications standards, making software upgrades of base stations a more attractive solution than the costly replacement of base stations\cite{ulversoy_software_2010}. \ac{sdr} also opens up the possibility of Cognitive Radios, a context sensitive radio system that can adapt depending on the radio channel conditions and applications. \\

A fundamental challenge of \ac{sdr} system is computational horsepower, because it needs to process complex data waveforms in a reasonable time-frame. Since \ac{sdr} involves transferring of signals and data from one system to another, this introduces considerable communication delays. Finally, general purpose processing systems introduces non-determinism in data processing and communication times.\\



Wireless devices share the wireless channel with other devices. Wireless protocol \ac{mac} layer is responsible for moderating access to the wireless channel. It typically uses \ac{tdma} and \ac{csma} to allocate the use of the channel. \ac{tdma} protocols schedule the allocation of the entire channel to one of the devices for a particular time duration. This requires global time synchronization among the devices so that the devices can understand when to transmit and receive. \ac{csma}, on the other hand uses the channel on an opportunistic basis, with the devices sensing if the channel is free or not. When it senses the channel to be free, it can start using it.\\

IEEE 802.15.4 \cite{noauthor_ieee_nodate} is a network specification deigned specially for \ac{LPWAN} i.e \ac{IOT} devices.
It specifically defines the Physical Layer and the \ac{mac} layer of the network stack.
With the aim to be enabler of \ac{sdr} testbeds for \ac{IOT} protocols, this project uses IEEE 802.15.4 based physical layer implementation for the evaluation of the \ac{sdr} platform. \\

LimeSDR \cite{noauthor_limesdr_nodate} is a new low-cost \ac{sdr} platform, which supports the desired frequency bands.
The lack of research on the characteristics of the platform made it the ideal choice for my \ac{sdr} platform. 


\section{Problem Context}

\subsection{CSMA}
As highlighted by \cite{schmid_experimental_2007}, \ac{sdr} based systems don't comply with the stringent timing constraints imposed by modern \ac{mac} protocols. Furthermore, the presence of long bus communication and processing delays create \textit{blind spots}\cite{schmid_experimental_2007} in carrier sensing. In fig \ref{blind_spots}, a packet is being transmitted through the air medium which is received by the \ac{sdr} system. Since there is communication and processing delays, the \ac{cpu} of the \ac{sdr} system  receives the packet completely at $t_1$ delayed from $t_0$ when the packet transfer ends on the air medium.

Once the packet has been received, the system wants to let the transmitting system about the successful reception by sending the \ac{ack} packet. It detects the medium is free using carrier sensing, but this information is actually past information that has been delayed by $t_1 - t_0$, hence the system is blind towards the real time channel situation when making the decision to transmit and might lead to a collision. \\ 

\begin{figure}[!h]
\centering
\includegraphics[width=0.6\textwidth]{Figure/BlindSpots.png}
\caption{Blind Spots Illustration(adapted from \cite{schmid_experimental_2007}).}
\label{blind_spots}
\end{figure}

Hence when designing \ac{mac} protocols, these delays needs to be taken into account to avoid collisions. This necessitates a closer understanding of these delays and how different system parameters affect these delays.

\subsection{TDMA}
\ac{tdma} based protocols are controlled by time slots, hence there is need for precise scheduling to ensure that the transmissions happen in the correct time-slot. The delays and imprecise scheduling can be tolerated by making the time-slots longer but that degrades the efficiency of the overall network. Modern contention based protocols(\ac{csma}) also require precise timing to implement inter-frame spacing.\\

Hence methods to implement precise time scheduling needs to studied.

\section{Research Questions}
\begin{itemize}
\item{ What are the timing delays in  LimeSDR based IEEE 802.15.4 network ?}
\begin{figure}[!h]
\centering
\includegraphics[width=0.75\textwidth]{Figure/RQ1.png}
\caption{Research Question 1.}
\label{rq1}
\end{figure}
\item{ How to implement precise scheduling in LimeSDR based IEEE 802.15.4 communication system? }
\end{itemize}

\subsection{Report Outline}
The remainder of the report is structured as follows.\textit{Chapter 2} introduces the previous work in this field, as well the needed background information on the Lime\ac{sdr} platform, the base system design and the relevant tools used in methods section. \textit{Chapter 3} introduces the experimental setup and the methods used in the measurement of the timing delays. It also discusses on methods for precise scheduling in LimeSDR based systems. \textit{Chapter 4} presents the experimental results, which are analyzed in \textit{Chapter 5}. Finally, \textit{Chapter 6} includes the concluding remarks and scope of future work.
